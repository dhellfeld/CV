\section{\small{RESEARCH EXPERIENCE}}

{\sl\bf NSSC Graduate Research Fellow} \hfill {\bf Aug 2015 - Present} \\
{\sl Nuclear Science and Security Consortium}, UC Berkeley \hfill {\sl Berkeley, CA} %\\
%University of California, Berkeley
\vspace{2pt}
\begin{itemize}[leftmargin=4ex] \itemsep -2pt
\item Analyzing the coded aperture imaging performance of a spherical active coded aperture CdZnTe-based imaging system.
\item Optimized the coded configuration of detectors using simulations and iterative techniques.
\item Currently investigating the feasibility of real-time 3D imaging and scene data fusion as the system is moved freely through an environment.   
%Perform research in collaboration with UC Berkeley and Lawrence Berkeley National Laboratory in 
\end{itemize} 

{\sl\bf Graduate Student Intern} \hfill {\bf June 2015 - July 2015} \\
{\sl Lawrence Livermore National Laboratory}, NACS Division \hfill {\sl Livermore, CA} \\
%Nuclear and Chemical Sciences Division, Rare Event Detection Group
Rare Event Detection Group
\vspace{2pt}
\begin{itemize}[leftmargin=4ex] \itemsep -2pt
%\item Performed Geant4 simulations and data analysis for the WATCHMAN project.
\item Investigated the impact of cosmogenic radionuclide backgrounds as a function of detector overburden and the effects on directional sensitivity.
\item Integrated reactor antineutrino-electron scattering event generator into the \href{https://github.com/rat-pac/rat-pac}{RAT-PAC} Geant4 simulation package.
%\item Completed and submitted a paper to a peer reviewed physics journal on nuclear reactor antineutrino directional sensitivity of water Cherenkov detectors.
\end{itemize} 

{\sl\bf NSSC Graduate Research Fellow} \hfill {\bf Nov 2014 - May 2015} \\
{\sl Nuclear Science and Security Consortium}, UC Berkeley \hfill {\sl Berkeley, CA} %\\
%University of California, Berkeley
\vspace{2pt}
\begin{itemize}[leftmargin=4ex] \itemsep -2pt
\item Analyzed antineutrino directional reconstruction and background suppression techniques.
\item Performed simulations and analysis of antineutrino scattering background sources in water.
%\item Performed master's thesis research on antineutrino detection for nuclear safeguards in collaboration with Lawrence Livermore National Laboratory.
\end{itemize} 
 
{\sl\bf Graduate Student Intern} \hfill {\bf June 2014 - Aug 2014} \\
{\sl Lawrence Livermore National Laboratory}, NACS Division \hfill {\sl Livermore, CA} \\
%Nuclear and Chemical Sciences Division, Rare Event Detection Group
Rare Event Detection Group
\vspace{2pt}
\begin{itemize}[leftmargin=4ex] \itemsep -2pt
\item Conducted Geant4 simulations and data analysis in ROOT for the proposed WATer CHerenkov Monitor of AntiNeutrinos (WATCHMAN) detector.
\item Studied the feasibility of nuclear reactor directionality using antineutrino-electron elastic scattering in WATCHMAN.
%\item Constructed an event generator in ROOT and incorporated the generator into two Geant4 simulations in order to simulate scattering events.
%\item Compared competing WATCHMAN simulation packages and presented findings in order to assist in the decision of which package should ultimately be used for the project.
\end{itemize}

{\sl\bf Graduate Research Assistant} \hfill {\bf Sep 2013 - Nov 2014} \\
{\sl Department of Nuclear Engineering}, Texas A\&M University \hfill {\sl College Station, TX} %\\
%Department of Nuclear Engineering
\vspace{2pt}
\begin{itemize}[leftmargin=4ex] \itemsep -2pt
\item Investigated the use of silicon photodiodes to enhance scintillation detection capabilities.
\item Designed and constructed a multi-detector (NaI) housing for a vehicle-mounted radiation detection system.
%\item Constructed and characterized a mobile vehicle-based radiation detection system for wide area monitoring in urban environments.
%\item Performed various research tasks in the fields of nuclear security and radiation detection instrumentation development.
%\item Assisted in the teaching and grading associated with the graduate radiation detection laboratory course.
\end{itemize} 