\section{\small{RESEARCH EXPERIENCE}}

\textbf{Senior Scientific Engineering Associate} \hfill \textbf{Aug 2019 - Present} \\
\textsl{Applied Nuclear Physics Group}, Lawrence Berkeley National Laboratory \hfill \textsl{Berkeley, CA} \\[-2.8ex]
\vspace{2pt}
\begin{itemize}[leftmargin=4ex] \itemsep -2pt
    \item Real-time quantitative 3D gamma-ray imaging and scene data fusion.
    \item Object detection and tracking with LiDAR.
    % \item Development of real-time 3D gamma-ray imaging and scene data fusion for sUAS swarms.
    % \item Quantitative imaging.
    % \item Something else here that will take up two lines of text, just like this sentence that I have written here. See, it is two lines long.
\end{itemize}

\textbf{Research Fellow} \hfill \textbf{Nov 2014 - Jul 2019} \\
\textsl{Nuclear Science and Security Consortium}, UC Berkeley \hfill \textsl{Berkeley, CA} \\[-2.8ex]
\vspace{2pt}
\begin{itemize}[leftmargin=4ex] \itemsep -2pt
    \item Modeling and imaging algorithm development for free-moving hand-held and UAS-mounted gamma-ray imagers (proximity, coded aperture, Compton).
    \item Experimental demonstration of omnidirectional 3D active coded mask imaging in real-time.
    \item Fusion of contextual sensors (e.g., LiDAR, RGB camera, IMU) and computer vision techniques (e.g., SLAM) with gamma-ray image reconstruction.
    % \item Use contextual sensors and computer vision techniques (e.g.~SLAM) to facilitate real-time 3D gamma-ray mapping.
    % \item Simulations in Geant4. Data analysis and image reconstruction in Python. Data fusion, communication, and control in ROS. GPU programming in OpenGL and OpenCL.
    % \item Aid in the design and characterization of a hand-held, CdZnTe-based, spherical active coded aperture gamma-ray imager (PRISM).
    % \item Modeling and algorithm development for the Portable Radiation Imaging Spectroscopy and Mapping (PRISM) detector - a hand-held CdZnTe-based active spherical coded aperture.
    % \item Optimized the coded configuration of PRISM using simulations and iterative techniques.
    % \item Experimentally demonstrated omnidirectional 2D coded aperture imaging and free-moving 3D coded aperture imaging with scene data fusion.
    % \item Analyzing the coded aperture imaging performance of a CdZnTe-based spherical active coded aperture gamma-ray imager.
    % \item Optimized the coded configuration of detectors using simulations and iterative techniques.
    % \item Currently investigating real-time free-moving 3D imaging with scene data fusion.
    % \item Perform research in collaboration with UC Berkeley and Lawrence Berkeley National Laboratory
\end{itemize}

\textbf{Physics Intern} \hfill \textbf{Jun - Aug 2015/2014} \\
\textsl{Rare Event Detection Group}, Lawrence Livermore National Laboratory \hfill \textsl{Livermore, CA} \\[-2.8ex]
\vspace{2pt}
\begin{itemize}[leftmargin=4ex] \itemsep -2pt
    \item Monte Carlo simulations and data analysis for a water Cherenkov antineutrino detector.
    \item Study on the feasibility of remote clandestine nuclear reactor directionality with antineutrino-electron elastic scattering.
    \item Investigation of potential electron scattering background sources in water and the impact of overburden, fiducial volume, and radon contamination on directionality.
    % \item Continued reactor antineutrino directionality work in the WATCHMAN collaboration.
    % \item Investigated the impact of cosmogenic radionuclide backgrounds on reactor antineutrino directionality in water as a function of detector overburden.
    %\ item Explored directional sensitivity as function of fiducial volume size and radon contamination.
    % \item Integrated reactor antineutrino-electron scattering event generator into the RAT-PAC Geant4 simulation package.
    % \item Integrated reactor antineutrino-electron scattering event generator into the \href{https://github.com/rat-pac/rat-pac}{RAT-PAC} Geant4 simulation package.
    % \item Completed and submitted a paper to a peer reviewed physics journal on nuclear reactor antineutrino directional sensitivity of water Cherenkov detectors.
    % \item Performed Geant4 simulations and data analysis in ROOT for the proposed WATer CHerenkov Monitor of AntiNeutrinos (WATCHMAN) detector.
    % \item Conducted preliminary studies on the feasibility of reactor directionality with antineutrino-electron elastic scattering in water.
    % \item Constructed an event generator in ROOT and incorporated the generator into two Geant4 simulations in order to simulate scattering events.
    % \item Compared competing WATCHMAN simulation packages and presented findings in order to assist in the decision of which package should ultimately be used for the project.
\end{itemize}

\textbf{Graduate Research Assistant} \hfill \textbf{Sep 2013 - Nov 2014} \\
\textsl{Department of Nuclear Engineering}, Texas A\&M University \hfill \textsl{College Station, TX} \\[-2.8ex]
\vspace{2pt}
\begin{itemize}[leftmargin=4ex] \itemsep -2pt
    \item Design, construction and characterization of a vehicle-mounted scintillator detector array for wide area radiological search in urban environments.
    \item Review on the use of solid-state photodiodes and photomultipliers in improving scintillation detection systems.
\end{itemize}
